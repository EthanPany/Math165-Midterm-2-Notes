\documentclass[oneside]{book}

\usepackage{amsmath, amsthm, amssymb, amsfonts}
\usepackage{thmtools}
\usepackage{graphicx}
\usepackage{setspace}
\usepackage{geometry}
\usepackage{float}
\usepackage{hyperref}
\usepackage[utf8]{inputenc}
\usepackage[english]{babel}
\usepackage{framed}
\usepackage[dvipsnames]{xcolor}
\usepackage{environ}
\usepackage{tcolorbox}
\tcbuselibrary{theorems,skins,breakable}

\setstretch{1.2}
\geometry{
    textheight=9in,
    textwidth=5.5in,
    top=1in,
    headheight=12pt,
    headsep=25pt,
    footskip=30pt
}

% Variables
\def\notetitle{MATH 165\\Linear Algebra \& Diff. Equation\\Midterm II \\Review Note with Examples}
\def\noteauthor{
    \textbf{Professor Madhu} \\ 
    {\LaTeX} by Ethan\\
    University of Rochester}
\def\notedate{Spring 2024}

% The theorem system and user-defined commands
\input{theorems.tex}
\input{commands.tex}

% ------------------------------------------------------------------------------

\begin{document}
\title{\textbf{
    \LARGE{\notetitle} \vspace*{10\baselineskip}}
    }
\author{\noteauthor}
\date{\notedate}

\maketitle
\newpage

\tableofcontents
\newpage

% ------------------------------------------------------------------------------
\chapter{Determinants}
\section{Lecture 9: Def. of Determinants \& it's calculation}





























\section{Lecture 10: Rank, Invertibility, Elementary Row Operations \& additional properties of determinants}







% ------------------------------------------------------------------------------
\chapter{Vector Spaces}
\section{Lecture 11: Vector Spaces, Zero-Vectors, Dimentions, basis, \& Linear Combinations}




\section{Lecture 12: Vector Subspace \& Proof}





\section{Lecture 13: Spanning Set}





\section{Lecture 14: Spanning, Linear Independence \& Basis}


\section{Lecture 15: Linear Independence \& Basis and dimension}


\section{Lecture 16: Basis, Row and Columns Spaces}































































\end{document}
